\documentclass[a4paper,12pt]{article}
    \usepackage{mathtools}
    \usepackage{amsmath}
    \usepackage{amssymb}
    \usepackage{fancyhdr}
    \usepackage{bm}
    \usepackage{enumitem}
    \pagestyle{fancy}
    \lhead{Midterm, Math M393, Winter 2020}
    \rhead{Eitan Seri-Levi}
    \setlength{\headheight}{16pt}
    \DeclarePairedDelimiter\abs{\lvert}{\rvert}
    \makeatletter
    \let\oldabs\abs
    \def\abs{\@ifstar{\oldabs}{\oldabs*}}
    \newcommand{\R}{\mathbb{R}}
    \renewcommand{\headrulewidth}{0.4pt}
    \renewcommand{\footrulewidth}{0.4pt}
    \begin{document}
    \begin{itemize}
      \item[1.] Construct a truth table for the following statement. $\lnot p \land q$.
      \begin{displaymath}
        \begin{array}{|c c c|c|}
        p &\lnot p & q  & \lnot p \land q\\
        \hline
        T & F & T & F\\
        T & F & F & F \\
        F & T & T & T\\
        F & T & F & F\\
        \end{array}
      \end{displaymath}  
      \item[2.] For propositions p and q, prove $p \rightarrow q$ is equivalent to $\lnot q \rightarrow \lnot p$ by applying a truth table. 
      \begin{displaymath}
        \begin{array}{|c c c c|c|c|}
        p  & q & \neg p & \neg q  & p \rightarrow q & \neg q \rightarrow \neg p\\
        \hline
        T & T & F & F & T & T\\
        T & F & F & T & F & F\\
        F & T & T & F & T & T \\
        F & F & T & T & T & T\\
        \end{array}
      \end{displaymath}   
      Therefore we have shown the two statements are equivalent
      \item[3.] Let $\mathbb{R}$ deonte the universe of real numbers. Determine if the following statemenet is true or false:
      \\
      $(\forall x)(\exists y)(x + y > 0)$
      \\ 
      The statement reads, for all $x$ there exists a $y$ such that $x+y > 0$. This statement is true, for all $x$ there does exist some $y$ value such that $x+ y>0$. 
      \item[4.] Suppose $x_1$ and $x_2$ are integers. Prove or provide a counterexample: If $x_1$ divides $x_2$ then $x_1$ divides $x_2(x_1+1)$
      \\
      Suppose $x_1$ divides $x_2$ there exists an integer $k$ such that $x_2 = kx_1$ and $x_2/x_1 = k$
      \\
      If $x_1$ divides $x_2(x_1+1)$
      \begin{align*}
        x_2(x_1+1) &= mx_1 \text{ for some integer m}\\
        x_2/x_1(x_1+1) &= m\\
        k(x_1 +1) &= m\\
        x_1 + 1 &= m/k\\
      \end{align*}
      Since $x_1 + 1$ is an integer, $m/k$ is also an integer. Therefore we have shown that $x_1$ divides $x_2(x_1+1)$
      \item[5.] Suppose a is a positive integer. Prove the following statement by contradiction: If $a^2$ is even then a is even.
      \\ 
      Proof by contradiction: Assume $a$ is a positive integer and $a^2$ is even, but $a$ is odd.
      \\
      Since $a$ is odd, $a=2k+1$ for some integer $k$. then
      \begin{align*}
        a^2 &= (2k+1)^2\\
        &= 4k^2 + 4k +1\\
        &=2(2k^2+2k)+1
      \end{align*}
      Let $m=2k^2+2k$
      \begin{align*}
        a^2 &=2(2k^2+2k)+1\\
        &= 2m + 1
      \end{align*}
      So by definition $a^2$ is odd. But this is impossible because $a^2$ is even, therefore if $a^2$ is even then $a$ is even.
      \item[6.] Provide either a proof or counterexample for the following statement:
      \\
      For all positive integers $n$, $n(n-2) \geq 0$
      \\
      Let $n=1$
      \begin{align*}
        1(1-2) = 1(-1) = -1
      \end{align*}
      Therefore the statement $n(n-2) \geq 0$ is false since $n(n-2)=-1$ when $n=1$
      \item[7.] Define $\mathbb{N}=\{1,2,3,...\}$ and $n\mathbb{Z}=\{...,-2n, -n, 0, n, 2n,...\}$ for $n \in \mathbb{N}$. Let $\mathbf{M}=\{n\mathbb{Z}: n \in \mathbb{N}\}$. Prove that $\bigcup \mathbf{M} \subseteq \mathbb{Z}$
      \\
      Inductive step:
      \\
      $M_1 \subseteq \mathbb{Z}$, Where $M_1 = \{\mathbb{Z}: n \in \mathbb{N}\}$ 
      \\
      Proof by induction:
      \\
      If $M_n \subseteq \mathbb{Z}$ then $M_{n+1} \subseteq \mathbb{Z}$
      \\
      $M_{n+1} = \{(n+1)\mathbb{Z}: n \in \mathbb{N}\}$
      \\
      $M_{n+1} = \{... , -2(n+1), -(n+1), 0, (n+1), 2(n+1), ...\}$
      \\
      Since $n \in \mathbb{N}$ by definition $n+1 \in \mathbb{N}$ since 1 is an integer.
      \\
      let $k=n+1$ then $M_{n+1}=\{...,-2k, -k, 0, k, 2k, ...\}$
      \\
      \\
      $M_{n+1}=\{k\mathbb{Z}: k \in \mathbb{N}\}$ and $M_{n+1}\subseteq \mathbb{Z}$
      Therefore $\bigcup \mathbf{M} \subseteq \mathbb{Z}$
    \end{itemize} 
\end{document}