\documentclass[a4paper,12pt]{article}
    \usepackage{mathtools}
    \usepackage{amsmath}
    \usepackage{amssymb}
    \usepackage{fancyhdr}
    \usepackage{bm}
    \usepackage{enumitem}
    \pagestyle{fancy}
    \lhead{Quiz 3, Math M393, Winter 2020}
    \rhead{Eitan Seri-Levi}
    \setlength{\headheight}{16pt}
    \DeclarePairedDelimiter\abs{\lvert}{\rvert}
    \makeatletter
    \let\oldabs\abs
    \def\abs{\@ifstar{\oldabs}{\oldabs*}}
    \newcommand{\R}{\mathbb{R}}
    \renewcommand{\headrulewidth}{0.4pt}
    \renewcommand{\footrulewidth}{0.4pt}
    \begin{document}
    \begin{itemize}
      \item[1.] Let $\mathbb{R}$ denote the universe of real numbers. Determine if the following statement is true or false:
      \[
        (\exists x)(\forall y)(x \leq y)
      \]
      This statement is true. There exists an x for all y such that x is less than or equal to y. That is, in the universe of real numbers, there will always exist a value x that is less than or equal to a value y. 

      \item[2.] Suppose $x_1$, $x_2$ and $x_3$ are integers. Prove if $x_1$ divides $x_2$, then $x_1$ divides $x_2x_3$
      \\
      Let $k$ be an integer and $x_1$ divides $x_2$ therefore $x_2 = kx_1$ for some integer $k$.
      \\
      To check if $x_1$ divides $x_2x_3$:
      \begin{align*}
        x_2x_3 &= (kx_1)x_3 \\
        &= x_1(kx_3)\\
      \end{align*}
      Therefore $kx_3$ must be an integer. We can say that $kx_3 = j$ where $j$ is an integer.
      \begin{align*}
        x_2x_3 &= jx_1 \\
        \frac{x_2x_3}{x_1} &= j
      \end{align*}
      Therefore $x_1$ divides $x_2x_3$
    \end{itemize} 
\end{document}