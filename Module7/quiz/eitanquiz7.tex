\documentclass[a4paper,12pt]{article}
    \usepackage{mathtools}
    \usepackage{amsmath}
    \usepackage{amssymb}
    \usepackage{fancyhdr}
    \usepackage{bm}
    \usepackage{enumitem}
    \pagestyle{fancy}
    \lhead{Quiz 7, Math M393, Winter 2020}
    \rhead{Eitan Seri-Levi}
    \setlength{\headheight}{16pt}
    \DeclarePairedDelimiter\abs{\lvert}{\rvert}
    \makeatletter
    \let\oldabs\abs
    \def\abs{\@ifstar{\oldabs}{\oldabs*}}
    \newcommand{\R}{\mathbb{R}}
    \renewcommand{\headrulewidth}{0.4pt}
    \renewcommand{\footrulewidth}{0.4pt}
    \begin{document}
    \begin{itemize}
      \item[1.] Use the Principal of Mathematical Induction to prove \\ $\sum^{n}_{i=1}2^i=2^{n+1}-2$
      \\
      When $n = 1$
      \begin{align*}
        \sum^{1}_{1}2^{i}&=2^{1+1}-2\\
        2 &= 2
      \end{align*}
      The statement holds for $n = 1$
      \\
      Using the Principal of Mathematical Induction, assume that for some $n \in \mathbb{N}$, $\sum^{n}_{i=1}2^i=2^{n+1}-2$ is a true statement. We will need to prove that the statement is true for $n+1$
      \begin{align*}
        \sum^{n+1}_{i=1}2^i&=2^{(n+1)+1}-2\\
      \end{align*}

      The equation on the left side looks like
      \begin{align*} 
        \sum^{n+1}_{i=1}2^i&=\sum^{n}_{i=1}2^{i} + 2^{n+1}
      \end{align*}

      Simplifying the equation on the right side yields
      \begin{align*}
        2^{(n+1)+1}-2&=2^{n+2}-2\\
      \end{align*}

      Check if the left hand and right hand side are equal
      \begin{align*}
        \sum^{n}_{i=1}2^{i} + 2^{n+1}&=2^{n+2}-2\\
        2^{n+1}-2 + 2^{n+1}&= 2^{n+2}-2\\
        2^{n+2}-2 &= 2^{n+2}-2
      \end{align*}
      Thus the statement is true for $n+1$. So by the Principal of Mathematical Induction, the statement is true for every $n \in \mathbb{N}$
      \item[2.] Let $f_i$ denote the $i^{th}$ Fibonacci number as defined in Section 2.5. Use the PCI to prove the following property of Fibonacci numbers: \\$f_{n+6}=4f_{n+3} + f_n$ for all $n \in \mathbb{N}$
      \\
      Consider a set S where $S = \{n \in \mathbb{N}: f_{n+6}=4f_{n+3} + f_{n}\}$
      \\

      When $n=1$
      \begin{align*}
        f_7 &= 4f_4 + f_1\\
        13 &= 13\\
      \end{align*}
      
      Therefore $1 \in S$
      \\
      When $n=2$
      \begin{align*}
        f_8 &= 4f_5 + f_1\\
        21 &= 21\\
      \end{align*}
      Therefore $2 \in S$
      \\
      Our inductive hypothesis:
      \\
      Suppose $m \geq 3$ and $\{1,2,3,...,m-1\}\subseteq S$
      \\
      We begin by applying the definition of Fibonacci numbers:
      \begin{align*}
        f_{m+6} &= f_{m+5} + f_{m+4}\\
        &= f_{m+4} + f_{m+4} + f_{m+3}\\
        &= 2(f_{m+4}) + f_{m+3}\\
        &= 2(f_{m+3} + f_{m+2}) + f_{m+3}\\
        &= 3(f_{m+3}) + 2(f_{m+2})\\
        &= 3(f_{m+3}) + 2(f_{m+1} + f_m)\\
        &= f_m + 3f_{m+3} + 2f_{m+1} + f_m\\
        &=f_m + 3f_{m+3} + f_{m+1} + f_{m+1} + f_{m}\\
        &= f_m + 3f_{m+3} + f_{m+1} + f_{m+2}\\
        &= f_m + 3f_{m+3} + f_{m+3}\\
        &= 4f_{m+3} + f_{m}
      \end{align*}
    Therefore $n \in S$ and $f_{n+6}=4f_{n+3} + f_n$ for all $n \in \mathbb{N}$
    \end{itemize} 
\end{document}